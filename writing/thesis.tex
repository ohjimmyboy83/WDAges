\documentclass[11pt,letterpaper,draft,titlepage]{article}
\usepackage[utf8]{inputenc}
\usepackage{amsmath}
\usepackage[margins=1in]{geometry}
\usepackage{amsfonts}
\usepackage{aas_macros}
\usepackage{amssymb}
\usepackage{makeidx}
\usepackage{graphicx}
\usepackage[english]{babel}
\usepackage{setspace}
\author{James Phillips}
\title{Age Gradients in Local White Dwarf Populations}
\doublespacing
\begin{document}
	\pagestyle{empty}

	\tableofcontents
	\newpage
	
	\pagenumbering{arabic}
	\pagestyle{plain}
	\section{Introduction}
		\subsection{Dating Stars is important}
		\paragraph{}
		Time presents a set of unique problems for the modern astronomer. Our view of the cosmos is a frozen tableau, distorted by the fun-house mirror of the finite speed of light. Although the objects in our sky develop and evolve like any other system, the sheer scale of the things often leave visible change too slow to observe in a human lifetime. Motion and the evolution brought by ongoing processes often must be inferred through physics; through physics astronomers have inferred a lot. 
		\paragraph{}
		Galaxies spin with stately grace within spectral halos of dark matter, all the while speeding ever faster away from each other. Clouds of cold hydrogen satisfy the Jeans Criterion and silently collapse in a motion at once inevitable as an avalanche yet imperceptible as a mountain's uplift. Stars burn their fuel at prodigious rates only dwarfed by the sheer mass of fuel they have left to burn. Even the great speed of eruptions at the end of stellar lifetimes is humbled by the great distances into which they expand.
		\paragraph{}
		It is as though we have caught the universe red-handed, and it has frozen in place as it tries to invent some excuse. 
		\paragraph{}
		How does galactic structure come to be and how does it evolve? To even begin to address this problem, an accurate determination of the ages of the most visible components of galaxies would be invaluable. Stars, in their formation and their emissions, in their violent deaths and procreation, drive galactic evolution in a fundamental way. Their ages and distributions can constrain, distinguish, and inform our models. 
		\paragraph{}
		It's a shame stellar age is so difficult to measure. 
		
		\subsection{Dating Stars is Hard}
		\paragraph{}
		Stars are frustratingly coy about their ages. The longest lived ones are the slowest to change, and the most dynamic ones represent a tiny minority. Ages must be inferred from measurable quantities, and no single indicator works for all stars, even restricted to the main sequence.\cite{Sod2010} Many dating techniques require special circumstances, like a large coeval sample or intensive, in-person field exploration of planetary bodies around that star. Many are only useful or sensitive for high-mass stars, stars with high inclination\cite{Bar2007},\cite{Sku1972}, or high-metallicity stars. Some require a "Solar Twin," in a narrow surface temperature range\cite{Car2016}.
		\paragraph{}
		Barnes \cite{Bar2007} described the qualities to look for in a good age indicator. A prospective quantity should be: (1) useful for individual stars, (2) have a "sensitive dependence on age," (3) as much insensitivity to other parameters as possible, (4) be able to be calibrated through comparison with objects of known age, (5) Invertibile, (6) posess known or calculable error margins, and (7) yield the same results within a sample of stars known to be coeval. The search for such indicators has met with some modest success, though age determinations of large populations of non-coeval stars remain elusive.
		
		\subsection{Empirical Methods}
		\paragraph{}
		Gyrochronology exploits the spin-down of a star throughout its lifetime to estimate the age. 
		
		
				
	\section{Methods}
		\subsection{Data Selection}
		\subsection{I basically plotted this crap}
		\subsection{you should give me a master's degree}
	
	\section{Conclusions}
	
	\bibliographystyle{alpha}
	\bibliography{thesis}
\end{document}
